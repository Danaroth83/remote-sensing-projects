\documentclass{article}

\usepackage{graphicx} % Required for inserting images
\usepackage{hyperref}
\usepackage{listings}

\title{Instructions to submit a project on git}
\author{Daniele Picone}
\date{June 5th, 2025}

\begin{document}

\maketitle

For the following operations we suggest to use a desktop computer from the university to make use of the fast speed connection.
Please operate exclusively on the C drive on Windows not to be limited in terms of data size. 

\section*{Creating a new remote repository}
\begin{itemize}
    \item Login, using your Agalan credential on the Gitlab of Gricad:
    \url{https://gricad-gitlab.univ-grenoble-alpes.fr/}
    \item Click on "New project" and then on "Create blank project"
    \item Assign a name to your project in the text box, select "Public", unselect "Add README.md" and finally click on "Create project"
    \item From now you have an assigned git repository with a name such as:\\
    https://gricad-gitlab.univ-grenoble-alpes.fr/username/project.git\\
    where \textrm{username} is your agalan username and \textrm{project} is the name that you have assigned to your project
\end{itemize}

\section*{Initializing a repository}

\begin{itemize}
    \item Install \textbf{git for Windows} from \url{https://gitforwindows.org/} and \textbf{git-lfs} from \url{https://git-lfs.com/}
    \item Launch the git CLI (CLI stands for command line interface) from the utility you have just installed. You will see a command terminal.
    \item From the command line interface, navigate to the root folder of your project:
    \begin{lstlisting}[language=bash]
        cd C:\pathtoproject        
    \end{lstlisting}
    \item If you already have initialized a git project, and you want to delete the previous content type:
    \begin{verbatim}
        rm -rf .git
    \end{verbatim}
    \item Initialize the project to connect to the remote git
    \begin{lstlisting}[language=bash]
      git init
      git remote add origin https://gricad-gitlab.univ-grenoble-alpes.fr/username/project.git
      git branch -M main
    \end{lstlisting}
    where \texttt{username} and \texttt{project} are the username and project name you have previously assigned.
\end{itemize}

\section*{First commit}

First, we need to configure your identity on your machine:
\begin{lstlisting}[language=bash]
   git config --global user.name "Your Name"
   git config --global user.email "you@grenoble-inp.fr"
\end{lstlisting}
where \texttt{Your Name} are your name and surname and \texttt{you@grenoble-inp.fr} is your institutional mail.

For a first commit, I suggest to create a \texttt{README.md} file to the root folder of your project and upload it remotely:
\begin{lstlisting}[language=bash]
    git add README.md
    git commit -m "Initial commit"
    git push origin main
\end{lstlisting}
It will ask you to type your Grenoble-INP username and password.
Verify that your readme file now shows on the Gricad page of your repository.

\section*{Commits with data content}

Identify what are all the extensions that are specific of those large files, such as \texttt{.pdf}, \texttt{.tif}, \texttt{.jpg}, etc.
You need to notify your repository that those files are large. For example for \texttt{tif} files type:
\begin{lstlisting}[language=bash]
    git-lfs track *.tif
    git-lfs track *.TIF
\end{lstlisting}
and repeat this operation for all extensions of files larger than 5MB.

In the second commit, I suggest to just add the code and the reports to the root folder of your project. If you have any data, please move it away to a temporary external folder outside of the root folder of your project.
Then type:
\begin{lstlisting}[language=bash]
    git status
\end{lstlisting}
to show all the files that you wanted to add to the online repository.
Then commit and pull once again:
\begin{lstlisting}[language=bash]
    git add .
    git commit -m "Intermediate commit"
    git push origin main
\end{lstlisting}

Finally, add your data back to the folder and try a final commit:

\begin{lstlisting}[language=bash]
    git add .
    git commit -m "Final commit"
    git push origin main
\end{lstlisting}

\section*{Failed commits}

In the case that Gitlab-Gricad does not allow you to push data, it is most likely the case that you are not properly tracking all the data content that is bigger than 5 MB. In that case I suggest to restart from the operation of "initializing a repository" and try again, being very careful to track all the large files when you type \texttt{git-lfs track *.pdf}, etc.

\end{document}

